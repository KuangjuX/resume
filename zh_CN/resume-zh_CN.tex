% !TEX TS-program = xelatex
% !TEX encoding = UTF-8 Unicode
% !Mode:: "TeX:UTF-8"

\documentclass{resume}
\usepackage{zh_CN-Adobefonts_external} % Simplified Chinese Support using external fonts (./fonts/zh_CN-Adobe/)
% \usepackage{NotoSansSC_external}
% \usepackage{NotoSerifCJKsc_external}
% \usepackage{zh_CN-Adobefonts_internal} % Simplified Chinese Support using system fonts
\usepackage{linespacing_fix} % disable extra space before next section
\usepackage{cite}

\begin{document}
\pagenumbering{gobble} % suppress displaying page number

\name{齐呈祥}

\basicInfo{
  \href{mailto:18630816527@163.com}{18630816527@163.com} $\bullet$
  \href{https://github.com/KuangjuX}{GitHub (\textbf{119} followers)} $\bullet$
  \href{https://www.linkedin.com/in/kuangjux}{linkedin}
}
 
\section{Education}
\datedsubsection{\textbf{\href{http://cic.tju.edu.cn/}{天津大学}}}{08/2019 -- 07/2023(Expected)}
计算机科学与技术, GPA: 3.361/4

\section{Selected Open-source Projects}
\datedsubsection{\href{https://github.com/Ko-oK-OS/xv6-rust}{\textbf{xv6-rust}}: 使用 Rust 语言重新实现与优化 xv6-riscv(\textbf{45} stars)}{03/2021 -- 08/2021}
\begin{itemize}
  \item 使用伙伴内存分配系统优化了内存分配系统
  \item 重新设计了 SpinLock/SleepLock,满足 RAII.
  \item 优化了文件系统,使之更加满足 Rust 的特性.
\end{itemize}

\datedsubsection{\href{https://github.com/KuangjuX/rCore-fat}{\textbf{rCore-fat}}: 拥有 FAT32 文件系统的 rCore-Tutorial-v3}{07/2021 - 08/2021}
\begin{itemize}
  \item 在 alloc 下实现了 FAT32 crate.
  \item 为 rCore-fat 重新设计了系统调用.
\end{itemize}

\datedsubsection{\href{https://github.com/KuangjuX/SimpleMIPS}{\textbf{SimpleMIPS}}: 一个简单的五级流水 mips cpu}{10/2021 - 12/2021}
\begin{itemize}
  \item 支持 57 条指令.
  \item 支持 SRAM 和 AXI 总线.
  \item 可运行在 GENESYS2 FPGA 上.
\end{itemize}


\datedsubsection{\href{https://github.com/KuangjuX/Trivial-TCP}{\textbf{Trivial-TCP}}: 使用 C 语言编写的 TCP 协议栈}{09/2021 - 10/2021}
\begin{itemize}
  \item 实现了 \textbf{连接管理、可靠传输、流量控制、拥塞控制} 等 TCP 标准
\end{itemize}

\datedsubsection{\href{https://github.com/KuangjuX/NEMU2020}{\textbf{NEMU}}: 使用 C 语言编写的 x86-32 模拟器(\textbf{10} stars)}{09/2020 - 01/2021}
\begin{itemize}
  \item 简单的调试器
  \item 指令译码、执行
  \item 分段、分页机制,二级 Cache
\end{itemize}

\section{Experience}
\datedsubsection{\textbf{\href{http://cic.tju.edu.cn/}{天津大学}}}{TianJin}
\datedsubsection{助教}{09/2021 -- 11/2021}
\begin{itemize}
  \item 2021 计算机系统实践
\end{itemize}

\datedsubsection{\textbf{\href{https://www.twt.edu.cn/home/}{天外天工作室}}}{TianJin}
\datedsubsection{Software Enginer}{09/2019 -- 01/2021}
\begin{itemize}
  \item 开发校务专区系统
  \item 维护党建系统
\end{itemize}


\section{Selected Awards}
\begin{itemize}
  \item \textbf{\href{https://os.educg.net/2021CSCC}{2021 全国大学生计算机系统能力大赛操作系统设计大赛}}\quad \textbf{功能赛道三等奖}
  \item \textbf{\href{ https://summer-ospp.ac.cn/ }{开源之夏 2021}}\quad \textbf{最佳质量奖(仅仅 5 位)}
\end{itemize}



\section{Skills}
\textbf{Programming Languages:} \small Rust, C, Go, Python, C++, Systemverilog/Verilog, PHP, JavaScript, HTML/CSS (按熟练度由高到低排序)

\textbf{Tools and Frameworks:} \small Git, GNU Make, CMake, QEMU, Vivado, VSCode, Gin, Vue, Laravel

\section{Others}
\begin{itemize}
  \item 自学了 CMU 15-445、MIT 6.S081 等国外公开课并且完成了对应的 lab assignments.
  \item 对操作系统、计算机体系结构、分布式系统感兴趣.
  \item 参与过 rcore-os、cs-self-learning 等开源社区的工作.
  \item 个人开源项目累计获得超过 100 stars.
\end{itemize}


\end{document}
