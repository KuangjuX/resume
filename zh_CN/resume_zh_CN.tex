\documentclass{cv}

\usepackage{fontawesome5}
\usepackage{fontspec}
\usepackage[colorlinks]{hyperref}
% ----------------------------------------------------------------------------- %

%\setmonofont{FiraCode}
\begin{document}
\name{齐呈祥}
\medskip

\basicinfo{
  \faPhone ~ (+86) 186-3081-6528
  \textperiodcentered\
  \faEnvelope ~ 18630816527@163.com
  \textperiodcentered\
  \faGithub ~ \href{https://github.com/KuangjuX}{\color{black}KuangjuX}
  \textperiodcentered\
  \faBlog ~ \href{http://blog.kuangjux.top}{\color{black}{blog}}
}

% ----------------------------------------------------------------------------- %

\section{教育背景}
\dateditem{\textbf{天津大学} \quad 计算机科学与技术\quad 本科}{2019年 -- 至今 }


% ----------------------------------------------------------------------------- %

\section{专业技能}
\smallskip
\textbf{Programming}
\quad C/C++/Rust/Go/Python/Js/PHP

\textbf{Tools} Git/GNU Make

% ----------------------------------------------------------------------------- %
\section{个人介绍}
天津大学计算机科学与技术三年级本科生,曾经做过前端开发与后端开发的工作,目前对操作系统和体系结构感兴趣,了解一些 RISC -V的架构,
喜欢Rust语言。


\section{工作经历}
\dateditem{\textbf{天津大学} \quad 助教} {2021年 8月 -至今}

\begin{itemize}
  \item 计算机系统实践课程
\end{itemize}

\dateditem{\textbf{天津大学天外天工作室} \quad 后端工程师} {2019年 9 月 - 2021年 2 月}

\begin{itemize}
  \item 开发校务管理平台
  \item 维护学校党建系统
\end{itemize}

% ----------------------------------------------------------------------------- %

\section{项目经历}

\datedproject{Ko-ok-OS/xv6-rust}{比赛项目}{2021年3月 - 至今}
\\
使用Rust语言重写和优化 MIT xv6-riscv 操作系统
\\
本次项目获得了 2021全国大学生计算机系统能力大赛 操作系统设计赛 功能设计赛道\textbf{三等奖}
\begin{itemize}
  \item 将 xv6-riscv 的内存分配系统使用伙伴内存分配系统实现
  \item 重新设计 SpinLock/SleepLock 使之作为智能指针使用,真正意义上实现了 RAII
  \item 重新优化了文件系统,尤其在inode和Buffer的实现上更加支持Rust特性
\end{itemize}


\datedproject{Ko-oK-OS/allocator}{比赛项目}{2021年3月 -- 2021年4月}
\\
基于Rust语言开发的伙伴内存分配系统,可以用于裸机环境下的内存分配器

\datedproject{LangHuan-Blessed-Land}{个人项目}{2021年1月 -- 2021年2月}
\\
基于Go语言编写的仿知乎后端程序
\begin{itemize}
  \item 支持Docker部署
  \item 基于Websocket构建,支持实时聊天功能
\end{itemize}

\datedproject{NEMU}{课内项目}{2020年9月 -- 2021年1月}
{\it 计算机系统实践项目}
\vspace{0.4ex}
\\ 基于C语言编写的x86-32的虚拟机
\begin{itemize}
  \item 实现了调试器
  \item 实现了指令的解析与执行
  \item 实现了分段、分页机制、二级缓存机制
\end{itemize}

\section{竞赛经历}
\textbf{2021操作系统大赛功能设计赛道}\quad \textbf{三等奖}

\end{document}